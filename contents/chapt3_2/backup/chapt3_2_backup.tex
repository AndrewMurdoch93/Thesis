\chapter{Policy gradient reinforcement learning}
\label{chp:rl}

In Chapter \ref{chp:mdps}, we introduced a tabular, value-based reinforcement learning technique called Sarsa.
In value-based techniques, the policy is derived directly from the action-value function $Q$, i.e. taking the argmax of the action-value function.
An argmax can only be used so long as the action space is discretised.
This is problematic in cases such as autonomous racing, where precise control actions are required. 

In this chapter, we introduce the Deep deterministic policy gradient (DDPG) and Twin delay deep deterministic policy gradient (TD3) methods.
These are two techniques that allow for continuous action spaces by representing the policy explicitly as a function.
We use a notation similar to Sutton and Barto \cite{sutton2020}, denoting a policy with parameters $\phi$ as $\pi_\phi$.
The parameters are learned based on the gradient of a scalar performance measure $J(\phi)$ with respect to the policy parameters.
Since these methods seek to maximise performance, \emph{gradient ascent} is performed in $J$:
\begin{equation}\label{eq:gradient_ascent}
    \phi_{t+1} =  \phi_{t} + \alpha \nabla J( \phi_{t} ),
\end{equation}
hence the name \emph{policy gradient}.
Equation \ref{eq:gradient_ascent} is analogous to Equation \ref{eq:gradient_descent} for gradient descent.
DDPG and TD3 are also known as \emph{actor-critic} techniques because their policies and value functions are referred to as actors and critics, respectively.


\section{Deep deterministic policy gradient}
The  DDPG algorithm proposed by Lillicrap et al. \cite{Lillicrap2016} is an \emph{off-policy} actor-critic method for continuous action spaces. 
We show the pseudocode for it in Algorithm \ref{alg:ddpg}. 
The fundamental concepts are the same as for Sarsa: (a) the agent selects an action according to the policy, (b) the policy is evaluated by updating the critic, and (c) the policy is improved by updating the actor.
However, the Lillicrap et al. \cite{Lillicrap2016} add two major components: (d) a replay buffer, and (e) target networks.

\textbf{Policy:}
As an off-policy algorithm, DDPG learns a desired policy from an exploration policy, where the exploration policy is used to interact with the environment.
This allows the agent to learn an optimal policy while exploring the environment.
The desired policy is a neural network that deterministically maps states to actions, $a_t = \pi_{\phi}(s_t)$, where $\phi$ are the neural network parameters. 
The exploration policy is then $\pi_{\phi}(s_t)$ with a noise process $\mathcal{N}_t$, 
\begin{equation}\label{eq:dpg_policy}
    a_t = \pi_{\phi}(s_t) + \mathcal{N}_t,
\end{equation}
where $\pi_{\phi}(s_t)$ is obtained by forward propagating $s_t$ through the network using Equations \ref{eq:forward_prop_1} to \ref{eq:forward_prop_3}.

\textbf{Replay buffer:}
DDPG makes use of an idea introduced by Lin \cite{lin1993}, the experience replay buffer. 
At each time step $t$, the agent's experience $e_t = (s_t,a_t,r_t,s_{t+1})$ is stored in a data set $\mathcal{B} = e_1, e_2, ... , e_N$. 
The update to the actor and critic is then applied to a batch of random samples from the replay buffer, $e \sim \mathcal{D}$. 
Maintaining a history of experiences allows each experience to potentially be used for several weight updates.
This results in greater data efficiency, reduces the variance of the updates by breaking correlations between samples, and prevents the neural network parameters from getting stuck in local minima \cite{mnih2013}.
In practice, replay buffers are implemented and used similar to the supervised learning dataset from equation \ref{eq:supdervised_dataset}.

\textbf{Target networks:}
Lillicrap et al. \cite{Lillicrap2016} employ a neural network as critic to predict the action-value function.
We denote the critic network with parameters $\theta$ as $Q_{\theta}$.
However, using the critic network to directly predict the target values during policy evaluation causes training instability, since the critic values are being updated towards a moving target \cite{mnih2013}.
Lillicrap et al. \cite{Lillicrap2016} address this issue by adapting the idea of \emph{target networks} from Mnih et al. \cite{mnih2013} for actor-critic methods.
The target networks are a pair of actor-critic networks, denoted $\pi_{\phi'}$ and $Q_{\theta'}$ respectively, which are only used to compute the target to the critic during policy evaluation.
The target network parameters $\phi'$ and $\theta'$ are updated towards $\phi$ and $\theta$ using a \emph{soft update} rule, 
\begin{equation}
\begin{split}
    \theta' &\leftarrow \tau \theta' + (1 - \tau) \theta, \\
    \phi' &\leftarrow \tau \phi + (1 - \tau) \phi',
\end{split}
\end{equation}
where $\tau \ll 1$ to prevent the target from changing too quickly, which in turn stabilises learning. 


\textbf{Policy evaluation:}
In the context of DDPG, the policy evaluation step is an update to the critic network $Q_{\theta}$ using gradient descent (Equation \ref{eq:gradient_descent}).
Furthermore, the cost function (Equation \ref{eq:cost_function}) is computed using the difference between the target and action-values evaluated by the updated critic over a batch of experiences sampled randomly from the replay buffer:
\begin{equation}
    J_{\theta}=\frac{1}{N} \sum_{i}^{N} (y_i - Q_{\theta}(s_i,a_i))^2,
\end{equation}
where $y_i$ is the target.
To find this target, the target actor and critics are used to apply the definition from Equation \ref{eq:sarsa_target} to every sample:
\begin{equation}
    y_i = r_i + \gamma Q_{\theta'}(s_{i+1}, \pi_{\theta'}(s_{i+1})).
\end{equation}
Performing a gradient descent step on a neural network is analogous to applying Equation \ref{eq:tab_policy_eval} to the tabular case.

\textbf{Policy improvement:}
In Sarsa, the policy was improved implicitly using an $\epsilon$-greedy policy. 
However, in the context of DDPG, the policy must be improved explicitly.
This is done by updating the neural network parameters with gradient ascent (Equation \ref{eq:gradient_ascent}).
The gradient of the performance measure $\nabla_{\phi}J(\phi)$ is derived by Silver et al. \cite{silver2014} as
\begin{equation}
\begin{split}
    \nabla_{\phi} J(\phi) &\approx \mathbb{E}[ \nabla_{\phi} Q_{\theta}(s,a) |_{s=s_t, a=\pi_{\phi}(s_t)}]\\
    &= \mathbb{E} [ \nabla_{a} Q_{\theta}(s,a)|_{s=s_t, a=\pi_{\phi}(s_t)} \nabla_{\phi} \pi_{\phi}(s)|_{s=s_t}].
    \label{eq:deterministic_policy_gradient}
    \end{split}
\end{equation}
Practically this is achieved by computing a gradient descent step (Equation \ref{eq:gradient_descent}) with a negative cost argument.
The cost function is set equal to the mean over the action-values of the experiences sampled for the policy evaluation step.

At the time of its release, DDPG was a state of the art model-free technique for continuous action spaces, even being competitive with algorithms which have full access to the dynamics model. 
However, there are improvements to be made.
Overestimation bias, whereby a state with a low value is approximated as having a high value is an issue identified by Fujimoto et al. \cite{Fujimoto2018}. 
The next section introduces an algorithm that makes improvements over DDPG by addressing value overestimation.

\begin{algorithm}[htb!]
\SetAlgoLined
\caption{Deep deterministic policy gradient}\label{alg:ddpg}
Initialise critic network $Q_{\theta}(s,a)$ and actor $\pi_{\phi}(s)$ with weights $\theta$ and $\phi$ \\
Initialise target network $Q_{\theta'}$ and $\pi_{\phi'}$ with weights $\theta' \leftarrow \theta$, $\phi' \leftarrow \phi$ \\
Initialise replay buffer $\mathcal{B}$ \\
\For{episode = 1, M}
{
    Initialise a random noise process $\mathcal{N}$ for action exploration \\
    Receive initial observation state $s_0$\\
    \For{t=0, T}
    {
        $\text{Select action } a_t = \pi_{\phi}(s_t) + \mathcal{N}_t$\\
        Execute action $a_t$, observe reward $r_t$, observe new state $s_{t+1}$ \\
        Store transition $(s_t, a_t, r_t, s_{t+1})$ in $\mathcal{B}$ \\
        Sample a random minibatch of $N$ transitions $(s_i, a_i, r_i, s_{i+1})$ \\
        \nonl from $\mathcal{B}$ \\
        Set TD target: $y_i = r_i + \gamma Q_{\theta'}(s_{i+1}, \pi_{\phi'}(s_{i+1}))$ \\
        Update critic by minimising the loss: \hspace{1.72cm}
            \rlap{\smash{$\left.\begin{array}{@{}c@{}}\\{}\\{}\end{array}\right\}%
            \begin{tabular}{l}Policy\\evaluation\end{tabular}$}} \\
        \nonl  $J_{\theta}=\frac{1}{N} \sum_{i}^{N} (y_i - Q_{\theta}(s_i,a_i))^2 $ \\
        Update the actor policy using the \\
        \nonl sampled policy gradient: \hspace{4cm}
            \rlap{\smash{$\left.\begin{array}{@{}c@{}}\\{}\\{}\end{array}\right\}%
            \begin{tabular}{l}Policy\\improvement\end{tabular}$}} \\
        \nonl $\nabla_{\phi} J \approx \frac{1}{N} \sum_{i}^{N} \nabla_{a} Q_{\theta}(s_i,a)|_{a=\pi_{\phi}(s_i)} \nabla_{\phi} \pi_{\phi}(s_i) $\\
        Update the target networks:\\
        $\theta' \leftarrow \tau \theta' + (1 - \tau) \theta$\\
        $\phi' \leftarrow \tau \phi + (1 - \tau) \phi'$\\
    }
}
\end{algorithm}


\section{Twin delay deep deterministic policy gradient}\label{sec:td3}

The twin delay deep deterministic policy gradient (TD3) presented by Fujimoto et al. \cite{Fujimoto2018} is a modification to DDPG that addresses overestimation of action-values as a result of using function approximators, as well as high variance in the critic update.
TD3 maintains the action selection mechanism, target networks and experience replay buffer from DDPG, but introduces three new ideas: target policy smoothing, clipped double Q-learning and delayed actor and target network updates.
The pseudocode is shown in Algorithm \ref{alg:td3}.



% Action-value overestimation is an issue since it may develop into a significant bias over many update steps, and may lead to poor quality policies \cite{hasselt2015}.
% Likewise, variance in the critic update produces poor quality policies and slows learning.
% TD3 maintains the target networks, experience replay buffer, action selection mechanism and actor update from DDPG, with a few key differences shown in Algorithm \ref{alg:td3}.

\begin{algorithm}[hbt!]
\caption{Twin delay deep deterministic policy gradient}\label{alg:td3}
Initialise critic networks $Q_{\theta_1}, Q_{\theta_2}$ and actor network $\pi_\phi$ 
with \\
\nonl parameters $\theta_1, \theta_2$ and $\pi_\phi$ \\
Initialise target networks $Q_{\theta_1'}$, $Q_{\theta_2'}$ and $\pi_{\phi'}$ with weights \\
\nonl $\theta_1' \leftarrow \theta_1$, $\theta_2' \leftarrow \theta_2$, and $\phi' \leftarrow \phi$ \\
Initialise experience replay buffer $\mathcal{B}$ \\
\For{episode = 1, M}
{

    \For{t=0, T}
    {
        Select an action with random exploration noise \\
        \nonl $a_t \sim \pi_{\phi}(s) + \epsilon$, $\epsilon \sim \mathcal{N}(0, \sigma)$ \\
        Execute action $a_t$, observe reward $r$ and new state $s_{t+1}$ \\
        Store transition tuple $(s_t,a_t,r_t,s_{t+1})$ in $\mathcal{B}$ \\
        Sample mini-batch of $N$ transitions $(s_i,a_i,r_i,s_{i+1})$ from $\mathcal{B}$ \\
        Select target actions:\\
        \nonl $\tilde{a} \leftarrow \pi_{\phi'}(s_{t+1}) + \epsilon$, $\epsilon \sim \text{clip}(\mathcal{N}(0,\tilde{\sigma}), -c,c)$ \\
        Set TD target:
        $y_i \leftarrow r_i + \gamma \min_{q=1,2} Q_{\theta_q'}(s_{i+1}, \tilde{a})$ \\
        Update critics by minimising the loss: \hspace{1.8cm}
            \rlap{\smash{$\left.\begin{array}{@{}c@{}}\\{}\\{}\end{array}\right\}
            \begin{tabular}{l}Policy\\evaluation\end{tabular}$}} \\
        \nonl $J_{\theta_q} \leftarrow  \frac{1}{N} \sum_{i}^{N} (y_i - Q_{\theta_q}(s_i,a_i))^{2} $ \\
        \If{$t \text{ mod } d$}
        {
            Update $\phi$ by the deterministic \\
            \nonl policy gradient: \hspace{5.2cm}
                \rlap{\smash{$\left.\begin{array}{@{}c@{}}\\{}\\{}\end{array}\right\}
                \begin{tabular}{l}Policy\\improvement\end{tabular}$}} \\
            \nonl $\nabla_\phi J(\phi) = \frac{1}{N} \sum_{i}^{N} \nabla_a Q_{\theta_1} (s_i,a) | _{a=\pi_{\phi}(s_i)} \nabla_\phi \pi_\phi(s_i)$ \\
            Update target networks: \\
            $\theta_{q}' \leftarrow \tau \theta_q + (1 - \tau) \theta_{q}'$ \\
            $\phi' \leftarrow \tau \phi + (1 - \tau) \theta'$ \\
        }
    }
}
\end{algorithm}

\textbf{Target policy regularisation:}
High variance critic updates in the policy evaluation step result in poor quality policies and slow learning.
Fujimoto et al. \cite{Fujimoto2018} address this issue by introducing a regularisation strategy for the critic update. 
They add small amount of noise $\mathcal{N}$ to the target action $\tilde{a}$,
\begin{equation}
    \tilde{a} \leftarrow \pi_{\phi'}(s_{t+1}) + \epsilon, \epsilon \sim \text{clip}(\mathcal{N}(0,\tilde{\sigma}), -c,c),
\end{equation}
where $c$ is the magnitude of upper and lower limits of the added noise.
This smooths the value estimate in the critic update.


\textbf{Clipped double Q-learning:}
Maximisation of a noisy value estimate in the policy evaluation step results in overestimation bias, whereby bad states are evaluated as high value. 
To combat overestimation bias, Fujimoto et al. \cite{Fujimoto2018} adapt double Q-learning from Hasselt et al. \cite{hasselt2015} for deterministic policy gradients. 
In double Q-learning, two critics, denoted $Q_{\theta_1}$ and $Q_{\theta_2}$ are maintained.
This allows two independent targets to be calculated at every time step.
Both critics are then updated towards the target containing the minimum of their two estimated action values:
\begin{equation}
    y_i \leftarrow r_i + \gamma \min_{q=1,2} Q_{\theta_q'}(s_{i+1}, \hat{a}).
\end{equation}
Furthermore, the update to the actor in the policy improvement step is always made with respect to $Q_{\theta_{1}}$.
Overall, the addition of double Q-learning stabilises and results in more reliable learning.

\textbf{Delayed updates}:
The final modification that TD3 makes is that the actor and target networks are only udpated every $d$ time steps.
This minimises the likelihood of repeating policy  updates with respect to an unchanged critic.

% To the best of our knowledge, TD3 is the state current of the art in reinforcement learning algorithms for continuous action spaces.


\section{Summary}
In this chapter, we have reviewed the DDPG and TD3 methods.
These RL techniques allow for continuous action-spaces by using a neural network to represent the policy.
This is especially important in the context of autonomous racing, as the driving task necessitates very fine control actions.
Furthermore, they generalise over states by using neural networks to approximate the action-value function.
To the best of our knowledge, TD3 is the state current of the art in reinforcement learning algorithms for continuous action spaces.

It was stated in Chapter \ref{chp:mdps} that solving a problem with RL requires that problem to be modelled as an MDP.
This will be the focus of the next chapter

% \newcolumntype{L}{>{\raggedleft\arraybackslash}p{6cm}}
\newcolumntype{M}{>{\centering\arraybackslash}p{2.8cm}}
\newcolumntype{R}{>{\centering\arraybackslash}p{1.8cm}}

\begin{table}[h]
\centering
\renewcommand{\arraystretch}{1.1}
\begin{tabularx}{0.85\textwidth} {LMR}
    \hline
    \textbf{Method} & \textbf{continuous action space} & \textbf{online} \\
    \hline
    Value-based, Monte Carlo  &  &   \\
    Value-based, temporal difference  &  & \checkmark  \\
    Policy-based & \checkmark &  \\
    Actor-critic & \checkmark & \checkmark \\
    \hline
\end{tabularx}
\caption[Criteria for reinforcement learning method selection]{A condensed decision criteria for reinforcement learning method selection.}
\label{table:rl_summary}
\end{table}

