\chapter{Reinforcement learning}\label{chp:rl}


Reinforcement learning (RL) is a paradigm of machine learning concerned with training an agent to take sequential decisions in an environment in order to maximise a reward signal.
This chapter begins with an introduction to Markov Decision Processes (MDPs), which provide the formal framework for modelling the sequential decision-making problem that reinforcement learning aims to solve.
A comparison of different reinforcment learning methods is then given.
After careful consideration, the twin delay deep deterministic policy gradient (TD3) RL method is chosen for implementation.
TD3 is particularly reliant on deep neural networks (DNNs). 
Consequently, the theoretical backgorund of DNNs are discussed, followed by an in-depth explanation of the TD3 method itself.


\section{Markov decision processes}\label{sec:mdps}

The Markov Decision process is a popular mathematical framework for modelling discrete-time decision-making processes where the outcomes are partly random and partly under the control of the \emph{agent} (i.e., the decision maker).
MDPs are useful for studying optimisation problems whereby the goal is for the agent to learn a sequence of actions that maximise a \emph{reward} signal.
Such optimisation problems are commonly found in the fields of robotics, automated control, and manufacturing \cite{White1985}.
The ability of MDPs to model sequential problems is useful for the autonomous racing problem, where the outcome of the race is determined by a sequence of control actions, and the exact vehicle dynamics model is often uncertain.



Within an MDP, the learner and decision maker is called an agent.
Everything outside of the agent constitutes the environment.
The agent and environment interact in a sequence of discrete time steps, $t=0,1,2,3, \cdots$, which is represented in Figure \ref{fig:agent_env_boundary}.
At every time step $t$, the agent receives a  representation of the environment known as the state (denoted as $S_t$), on which basis it selects an action $A_t$.
Furthermore, the environment represents a set of state transition probabilities that are dependent only on the previous state and action.
At the next time step $t+1$, the agent receives a scalar reward $R_{t+1}$, along with a new state $S_{t+1}$, which is sampled from the environment \cite{sutton2020}.
This agent-environment interaction is repeated, giving rise to a trajectory
\begin{equation}\label{eq:mdp_trajectory}
S_0, A_0, R_1, S_1, A_1, R_2, S_2, A_2, R_3, \cdots.
\end{equation}


\begin{figure}[!htb]
\centering
\input{contents/chapt3/figs/agent_environment_boundary.pdf_tex}
\caption[The agent environment interaction within an MDP]{The agent environment interaction within an MDP. Adapted from Sutton and Barto \cite{sutton2020}. }
\label{fig:agent_env_boundary}
\end{figure}


Within the MDP framework, the objective of the agent is described in terms of the reward signal: it must formulate a policy $\pi(a|s)$ that maximises the sum of discounted rewards over the course of its entire trajectory.
This sum of discounted rewards is called the \emph{return}, and is denoted $G_t$,
\begin{equation}
G_t \doteq R_{t+1} + \gamma R_{t+2} + \gamma^2 R_{t+3} + ... = \sum_{k=t+1}^{T} \gamma^{k-t-1} R_{k},
\label{eq:G_t_discount}
\end{equation}
where $T$ denotes the final time step in the trajectory, and $\gamma$ is a discount factor that determines the weighting of future rewards \cite{sutton2020}.















%%%%%%%%%%%%%%%%%%%%%%%%%%%%%%%%%%%%%%%%%%%%%%%%%%%%%%%%%%%%%%%%%%%%%%%%%%%%%%%%%%%%%%%%%%%%%%%%%%%%%%%%%%%%%%%%%%%%%%%%%%%%%%%%%%%%

\section{Taxonomy of reinforcement learning}

A taxonomy of RL algorithms is shown in Figure \ref{fig:RL:taxonomy}.
Broadly speaking, there are two classes of RL techniques; model-based and model free.
Model-free RL methods aim to learn a policy solely through direct trial-and-error interactions with the environment, without utilising a representation of the environment dynamics.
On the other hand, model-based reinforcement learning involves learning a model of the environment dynamics, which the agent then uses to plan ahead before taking an action \cite{wang2019benchmarking}.
Due to the complexity involved in learning the dynamics of a race-car, we opted to limit the scope of the project to encompass only model-free methods.

Within model-free techniques, a relevant distinction that can be made is that of value-based and policy gradient methods.
Value-based RL methods  focus on estimating the action-value function, which represents the expected return of being in a particular state taking a specific action.
In these methods, the policy is derived from the action-value function.
For instance, a greedy policy always selects the action associated with the highest action-value.
However, a limitation of value-based methods is that they require the action-space to be discretised \cite{sutton2020}.
In the context of autonomous racing, where precise control actions are necessary for maintaining vehicle safety, discretized action-spaces fall short in providing the required level of fidelity for control.

Meanwhile, policy gradient techniques form a class of RL methods that explicitly represent the agent's policy as a function.
These methods directly optimise the policy, which serves as a mapping between states and actions, in order to maximise performance.
Representing the policy as a differentiable function such as a DNN allows for the representation of continuous state and action spaces \cite{silver2014}.
We have therefore chosen to employ policy gradient methods in this study.
Specifically, we use a state of the art policy gradient method known as twin delay deep deterministic policy gradient (TD3) \cite{Fujimoto2018}.
Due to this algorithm's reliance on DNNs to represent the policy, we briefly discuss DNNs before detailing the algorithm itself.


\begin{figure}[htb!]
    \centering
    \input contents/chapt3/figs/RL_taxonomy.tex
    \caption[A taxonomy of reinforcement learning algorithms]{A taxonomy of reinforcement learning algorithms.}
    \label{fig:RL:taxonomy}
\end{figure}










%%%%%%%%%%%%%%%%%%%%%%%%%%%%%%%%%%%%%%%%%%%%%%%%%%%%%%%%%%%%%%%%%%%%%%%%%%%%%%%%%%%%%%%%%%%%%%%%%%%%%%%%%%%%%%%%%%%%%%%%%%%%%%%%%%%%%%%%%%%%%%%%%%%%%%

\section{Deep neural networks}\label{sec:dnns}

% According to Goodfellow \cite{Goodfellow2016},

Deep neural networks (DNNs) are computational models which can represent complex relationships in data.
We present the theory surrounding DNNs in the context of supervised learning.
In supervised learning, the goal is to approximate a functional mapping, which is denoted as $y=f(\bm{x})$.
However, the exact function $f(\bm{x})$ is unknown,  and instead, we have a data set
\begin{equation}\label{eq:supdervised_dataset}
    \mathcal{D} = \{ (\bm{x}^{[1]}, y^{[1]}), (\bm{x}^{[2]}, y^{[2]}), \cdots, (\bm{x}^{[n]}, y^{[n]}) \}
\end{equation}
consisting of examples where $f(\bm{x})$ was evaluated at inputs $\bm{x}^{[i]}$ to produce corresponding outputs $y^{[i]}$.
The goal of an DNN is to define an mapping $\hat{y}=\hat{f}(\bm{x},\bm{\theta})$ which approximates $f(\bm{x})$ with parameters $\bm{\theta}$.
The values of $\bm{\theta}$ that result in the best approximation are learned using the provided dataset examples \cite{Goodfellow2016}.
Furthermore, in the context of applying DNNs to the TD3 algorithm, we are particularly interested in a subset of supervised learning known as regression, whereby both the input and output variables are continuous.


Taking inspiration from neural networks in the human brain, the basic component comprising DNNs are \emph{artificial neurons}, which are analogous to their biological counterpart.
These artificial neurons apply a non-linear function their inputs, and are organised in layers.
We focus our discussion on feedforward neural networks (FNNs), which are a type of DNN whereby information is passed through the layers of the DNN in sequential order.
Furthermore, we limit the discussion to \emph{fully connected layers}, in which each of the inputs of neurons in a given layer are connected to he outputs of each of the neurons through a weight, denoted with $w$.
An example of a feedforward DNN with fully connected layers is shown in Figure \ref{fig:neural_network}.
According to Ng et al. \cite{Ng2019}, the weights connecting each neuron in the $k$th layer to the preceding layer is given by the matrix
\begin{equation}
    \mathbf{W}^{(k)} = 
    \begin{bmatrix}
        w_{11} & \cdots &  w_{1d}   \\
        \vdots & \ddots &           \\
        w_{m1} &        & w_{md}    \\
    \end{bmatrix}, 
\end{equation}
where $m$ is the number of neurons in the $k$th layer, and $d$ is the number inputs to each neuron (i.e., the number of neurons in the preceding layer).
Furthermore, each neuron takes a bias, denoted with $b$, as input. 
The biases of neurons in the $k$th layer are represented as a vector
\begin{equation}
    \mathbf{b}^{(k)} = [ b_1 \ldots b_m ]^\intercal,
\end{equation}
The weights and biases of all the layers in the network make up its parameters, and are denoted with $\theta$.

\begin{figure}[htb!]
    \centering
    \input contents/chapt3/figs/neural_network.tex
    \caption[A feed forward neural network]{
    A three layer feedforward DNN with three fully connected layers: an input and a hidden layer of three units each, and an output layer of one unit. 
    The weight and bias connections of the $k$th layer are denoted $\mathbf{W}^{(k)}$ and $\mathbf{b}^{(k)}$ respectively.
    Furthermore, the DNN receives an input vector $\bm{x}=[x_1, \ldots ,x_d]$, and output $\hat{y}$.}
    \label{fig:neural_network}
\end{figure}


% \subsection{Forward propagation}

% The process of evaluating an input $\bm{x}$ to generate an output $\hat{y} = \hat{f}(\bm{x},\theta)$, is known as forward propagation.
% In forward propagation for feedforward DNNs, the output of each layer is computed sequentially from first to last \cite{Goodfellow2016}.
% The output of the $k$th layer is denoted with $\mathbf{h}^{(k)}$.

The inputs to the $k$th layer of a DNN are processed in two steps to produce an output. First, a \emph{pre-activation} function $\bm{a}^{(k)}(\bm{x})$ 
adds the biases and product of the weight matrix and the vector output of the preceding layer (denoted as $\mathbf{h}^{(k-1)})$:
\begin{equation}\label{eq:dnn_preact}
    \bm{a}(\bm{x}) = \mathbf{b}^{(1)} + \mathbf{W}^{(k)} \mathbf{h}^{(k-1)}(\bm{x}),
\end{equation}
The output of the $k$th layer is then calculated by applying an \emph{activation} function, denoted as $g^{k}(a)$, element-wise to the result of the pre-activation from Equation \ref{eq:dnn_preact}:
\begin{equation}\label{eq:dnn_act}
    \mathbf{h}^{(k)}(\bm{x}) = \bm{g}^{(k)}( \bm{a}^{(k)} (\bm{x}) ).
\end{equation}
Popular choices for activation functions include a linear function whose output is identical to its input
\begin{equation}
    g(a) = a,
    \label{eq:linear_act}
\end{equation}
the rectified linear unit (ReLU), which takes the maximum between zero and the weighted sum of its inputs
\begin{equation}
    g(a) = \max(0,a),
    \label{eq:ReLU_act}
\end{equation}
as well as the hyperbolic tangent
\begin{equation}
    g(a) = \tanh(a).
    \label{eq:tanh_act}
\end{equation}


The process of evaluating an input $\bm{x}$ to generate an output $\hat{y} = \hat{f}(\bm{x},\theta)$ is known as forward propagation.
During forward propagation, Equations \ref{eq:dnn_preact} and \ref{eq:dnn_act} are applied to every layer sequentially from first to last \cite{Goodfellow2016}, 
as shown in Algorithm \ref{alg:forward_prop}.
In this algorithm, the input to the DNN is set as the 0th layer of the network, and the output of the DNN corresponds to the output of the final layer.

\begin{algorithm}[htb!]
\caption[Evaluating a single example input via forward propagation]{Evaluating a single example input $\bm{x}$ via forward propagation. Adapted from Goodfellow et al. \cite{Goodfellow2016}.}
\label{alg:forward_prop}
\nonl\textbf{Input}: An input $\bm{x}$, DNN with parameters $\{\mathbf{W,b}\} \in \bm{\theta}$ \\
\nonl\textbf{Output}: DNN output $\hat{y}$ \\
\vspace{0.3cm} 
$\mathbf{h}^{(0)} \leftarrow \bm{x}$ \\
\For{$k=1$, $\cdots$, $K$}
{
    $\mathbf{a}^{(k)} \leftarrow \mathbf{b}^{(k)} + \mathbf{W}^{(k)} \mathbf{h}^{(k-1)}$ \\
    $\mathbf{h}^{(k)} \leftarrow \mathbf{g}^{(k)}(\mathbf{a}^{(k)})$ \\
}
$\hat{y} \leftarrow \mathbf{h}^{(K)}$ \\
\end{algorithm}



\subsection{Gradient-based learning}\label{sec:grad_des}
The process through which the optimal set of parameters are determined such that $\hat{f}(\bm{x}, \bm{\theta})$ approximates the mapping between 
input and output data as accurately as possible is called training.
Training is accomplished by incrementally adjusting the parameters of the DNN, such that a \emph{cost function}, denoted as $J$, is minimised. 

The cost function measures the error between the DNN predicted output $\hat{y} = \hat{f}(\bm{x}, \bm{\theta})$ and target examples of the true function $y = f(\bm{x})$. 
A popular cost  function that we consider in this study is the mean squared error (MSE) over a batch of data samples,
\begin{equation}
    J( \bm{y}, \bm{x}, \bm{\theta}) = \frac{1}{N} \sum_{i=1}^{N} (y^i - \hat{f}(\bm{x}^i, \bm{\theta}))^2,
    \label{eq:cost_function}
\end{equation}
where $\bm{y} = [ y^1, y^2, \ldots , y^N ]$ is a vector of training data (often referred to as the \emph{target}), 
and $\bm{\hat{x}} = [ x^1, x^2, \ldots , x^N ]$ is a vector of input values.
When this cost function is minimised over all of the available training data, the neural network becomes a better approximator of the true function $f(\bm{x})$.


The iterative process of adjusting the DNN parameters such that the cost function decreases is known as gradient descent.
% In gradient descent, the parameters of the DNN at a given time step $t$, denoted as $\bm{\theta}_t$, undergo updates by computing the partial derivative of the cost function with respect to those parameters. 
In gradient descent,  the partial derivative of the cost function with respect to the DNN parameters are computed at every time step. 
The parameters are then adjusted slightly so that the cost decreases, such that the parameters at the next time step are
\begin{equation}
    \bm{\theta}_{t+1} = \bm{\theta}_{t} - \alpha \nabla_{\bm{\theta}_t}J(\bm{\theta}_t),
    \label{eq:gradient_descent}
\end{equation}
where $\alpha$ is a scalar controlling the magnitude of the updates, and is known as the learning rate.

A popular gradient descent method that we consider in this study, which is based on Equation \ref{eq:gradient_descent}, is adaptive moment estimation (Adam), introduced by Kingma and Ba \cite{kingma2015}. 
Adam computes adaptive learning rates for each parameter in the DNN by keeping track of a decaying average of past gradients, referred to as the momentum. 












%%%%%%%%%%%%%%%%%%%%%%%%%%%%%%%%%%%%%%%%%%%%%%%%%%%%%%%%%%%%%%%%%%%%%%%%%%%%%%%%%%%%%%%%%%%%%%%%%%%%%%%%%%%%%%%%%%%%%%%%%%%%%%%%%%%%%%%%%%%%%%%%%%%%%%

\section{Twin delay deep deterministic policy gradient}\label{sec:td3}

The twin delay deep deterministic policy gradient (TD3) algorithm by Fujimoto et al. \cite{Fujimoto2018}, is a popular RL technique applied to solve robotics problems due to its capability to handle continuous action-spaces.
In TD3, the policy, which is denoted as $\pi_{\bm{\phi}}(s)$ and commonly known as the actor, is represented as a DNN with parameters $\bm{\phi}$.
To update the actor parameters such that the policy improves, gradient ascent is performed with respect to a performance measure $J(\bm{\phi})$,
\begin{equation}\label{eq:policy_gradient_ascent}
    \bm{\phi}_{t+1} = \bm{\phi}_{t} + \alpha \nabla_{\bm{\phi}} J(\bm{\phi}_t).
\end{equation}
The performance measure $J$ is defined in terms of an \emph{action-value function} (denoted $Q_{\bm{\theta}}(s,a)$), 
which is the expected return $\mathbb{E}[G_t]$ from the current state to the end of the episode, 
given that a specific action is selected in the current state, after which the policy $\pi_{\bm{\phi}}$ is followed.
This action-value function is known as a critic, and is represented by a DNN with parameters $\bm{\theta}$.
In TD3, the idea is to set the performance measure equal to the action-value function
\begin{equation}
    J(\bm{\phi}) = Q_{\bm{\theta}}(s,a)| _{a=\pi_{\bm{\phi}}(s)}.
\end{equation}
Updating the actor parameters as in Equation \ref{eq:policy_gradient_ascent} will then result in a policy that maximises the expected return, thus solving the MDP. 
TD3 is presented in Algorithm \ref{alg:td3}.


\begin{algorithm}[hbt!]
\caption[Twin delay deep deterministic policy gradient (TD3)]{The twin delay deep deterministic policy gradient (TD3) algorithm, adapted from Fujimoto et al. \cite{Fujimoto2018}.}\label{alg:td3}
Initialise critic networks $Q_{\bm{\theta}_1}, Q_{\bm{\theta}_2}$ and actor network $\pi_{\bm{\phi}}$ 
with \\
\nonl parameters $\bm{\theta}_1, \bm{\theta}_2$ and $\bm{\phi}$ \\
Initialise target networks $Q_{\bm{\theta}_1'}$, $Q_{\bm{\theta}_2'}$ and $\pi_{\bm{\phi}'}$ with weights \\
\nonl $\bm{\theta}_1' \leftarrow \bm{\theta}_1$, $\bm{\theta}_2' \leftarrow \bm{\theta}_2$, and $\bm{\phi}' \leftarrow \bm{\phi}$ \\
Initialise experience replay buffer $\mathcal{B}$ \\
\vspace{0.5cm}
\For{\emph{episode} = 0, M}
{
    \For{t=0, T}
    {
        Select an action with random exploration noise \\
        \nonl $a_t \sim \pi_{\bm{\phi}}(s) + \epsilon$, $\epsilon \sim \mathcal{N}(0, \sigma)$ \\
        Execute action $a_t$, observe reward $r$ and new state $s_{t+1}$ \\
        Store transition tuple $(s_t,a_t,r_t,s_{t+1})$ in $\mathcal{B}$ \\
        \vspace{0.5cm}
        Sample mini-batch of $N$ transitions $(s_i,a_i,r_i,s_{i+1})$ from $\mathcal{B}$ \\
        Select target actions for every sample:\\
        \nonl $\tilde{a} \leftarrow \pi_{\bm{\phi}'}(s_{i+1}) + \epsilon$, $\epsilon \sim \text{clip}(\mathcal{N}(0,\tilde{\sigma}), -c,c)$ \\
        Set TD target:
        $y \leftarrow r_i + \gamma \min_{q=1,2} Q_{\bm{\theta}_q'}(s_{i+1}, \tilde{a})$ \\
        Update critics by minimising the loss: %\hspace{1.8cm}
            % \rlap{\smash{$\left.\begin{array}{@{}c@{}}\\{}\\{}\end{array}\right\}
            % \begin{tabular}{l}Policy\\evaluation\end{tabular}$}} \\
        $\bm{\theta}_q \leftarrow \text{argmin}_{\bm{\theta}_q} ( \frac{1}{N} \sum_{i=0}^{N} (y_i - Q_{\bm{\theta}_q}(s_i,a_i))^{2} ) $ \\
        \vspace{0.5cm}
        \If{$t$ \emph{mod} $d$}
        {
            Update $\bm{\phi}$ by the deterministic policy gradient: \\ %\hspace{5.2cm} 
                % \rlap{\smash{$\left.\begin{array}{@{}c@{}}\\{}\\{}\end{array}\right\}
                % \begin{tabular}{l}Policy\\improvement\end{tabular}$}} \\
            \nonl $J(\bm{\phi}) = \frac{1}{N} \sum Q_{\bm{\theta}_1}(s_i,a) | _{a=\pi_{\bm{\phi}}(s_i)}$ \\
            Update target networks: \\
            $\bm{\theta}_{q}' \leftarrow \tau \bm{\theta}_q + (1 - \tau) \bm{\theta}_{q}'$ \\
            $\bm{\phi}' \leftarrow \tau \bm{\phi} + (1 - \tau) \bm{\theta}'$ \\
        }
    }
}
\end{algorithm}



In the first two lines of this algorithm, the actor and two critic DNNs are initialised as $\pi_{\bm{\phi}}$ and $Q_{\bm{\theta}_q}$, respectively.
The actor is parameterised by $\bm{\phi}$, and the critics by $\bm{\theta}_q$, where the subscript $q$ indicates the critic number.
Two critics are utilised to combat a phenomenon known as \emph{overestimation bias}, whereby states with a low real return are evaluated as having a high expected return. 


Subsequently, \emph{target networks} are initialised for each critic and the actor.
These target networks are identical in structure to their corresponding counterparts, and are indicated with a prime symbol.
Therefore, the target actor and critics are denoted as $\pi_{\bm{\phi}}'$ and $Q_{\bm{\theta}_q}'$, respectively.
These target networks are updated at a slower rate than their counterparts.
According to Mnih et al. \cite{mnih2013}, utilising these target networks to update the action-value function increases training stability.


Next, a replay buffer, denoted by $\mathcal{B}$, is initialised as an empty array to store the agents experiences.
These experiences are stored as a tuple 
\begin{equation}
    \bm{e}_t = (s_t,a_t,r_t,s_{t+1}),
\end{equation}
where $t$ is the time step of the experience.
By maintaining a history of experiences, each experience can be sampled multiple times to update the parameters of the actor and critics.
This approach also helps to break correlations between samples, preventing the DNN parameters from converging to a sub-optimal local minima \cite{mnih2013}.


After initialising the actor, critics, and replay buffer, the TD3 algorithm executes for $M$ number of of episodes.
Each of these episodes represents a trajectory of state, action and reward sequences, as shown in Equation \ref{eq:mdp_trajectory}.
At each step of the episode, the algorithm samples an action from the agent, after which the environment provides the next state and reward.
Subsequently, the critics are updated to improve the accuracy of action-value estimation.
Additionally, the policy and target networks are updated every $d$ time steps.
These processes are now explained in more detail.


In line $6$, an action is sampled from the actor using
\begin{equation}\label{eq:dpg_policy}
    a_t = \pi_{\bm{\phi}}(s_t) + \epsilon, \hspace{0.3cm} \epsilon \sim \mathcal{N}(0,\sigma),
\end{equation}
where $ \pi_{\bm{\phi}}(s_t)$ is obtained by forward passing the state through the actor DNN.
To encourage exploration, Gaussian noise with a mean of $0$ and a standard deviation of $\sigma$ is added to each action.
After sampling the new state , denoted as $s_{t+1}$ and reward, denoted as $r_{t}$ from the environment in line $7$, the agent's experience tuple is stored in the replay buffer in line $8$.



The process of updating the critics occurs in lines $9$ to $12$. 
Firstly, a batch of $N$ experiences is sampled from the replay buffer.
Each experience is represented as a tuple comprised of a state $s_i$, action $a_i$, reward $r_i$ and subsequent state $s_{i+1}$.
For each of these experiences, the target actor is used to select \emph{target actions} associated with the subsequent state,
\begin{equation}
    \tilde{a} \leftarrow \pi_{\bm{\phi}'}(s_{i+1}) + \epsilon, \epsilon \sim \text{clip}(\mathcal{N}(0,\tilde{\sigma}), -c,c),
\end{equation}
after which Guassian noise is added to the target action with upper and lower magnitude bounds of $c$.
The critic DNNs are then updated towards a \emph{TD target} vector, denoted as $y$, in line $12$.
Each element in the TD target vector corresponds to a sampled experience $i$, and is computed by
\begin{equation}
    y_i \leftarrow r_i + \gamma \min_{q=1,2} Q_{\bm{\theta}_q'}(s_{i+1}, \tilde{a}).
\end{equation}
The cost to be utilised in the gradient descent step for the critics, which we denote as $J(\bm{\theta})$, is computed as the MSE between the TD target and the action-values of the sampled experiences, as evaluated by the critics themselves:
\begin{equation}
     J(\bm{\phi}) = \frac{1}{N} \sum_{i=0}^{N} (y_i - Q_{\bm{\theta}_q}(s_i,a_i))^{2}.
\end{equation}
Note that the cost is computed twice by taking the MSE between the target and both critics.
The parameters of both critics are then updated using gradient descent with the objective of minimizing the cost function represented in line $12$.
Utilising two critics in this manner aims to overcome the issue of overestimation bias of action-values.


The updates to the actor and target DNNs occur every $d$ time steps to reduce the variance of their updates. 
The update to the actor parameters is performed by gradient ascent with respect to the performance measure $J(\bm{\phi})$.
This performance measure is defined as the the average of the first critics evaluation of the sampled states, where the action is given by the current policy $\pi_{\bm{\theta}}$, over the $N$ experiences sampled from the replay buffer:
\begin{equation}
    J(\bm{\theta}) = \frac{1}{N} \sum Q_{\bm{\theta}_1}(s_i,a) | _{a=\pi_{\bm{\phi}}(s_i)}.
\end{equation}
Therefore, updating the policy maximizes the performance measure, as well as the estimated return for each action.


Lastly, in lines 15 to 17, the target network parameters $\bm{\phi}'$ and $\bm{\theta}'$ are updated towards $\bm{\phi}$ and $\bm{\theta}$ using a \emph{soft update} rule, 
\begin{equation}
\begin{split}
    \bm{\theta}' &\leftarrow \tau \bm{\theta}' + (1 - \tau) \bm{\theta}, \\
    \bm{\phi}' &\leftarrow \tau \bm{\phi} + (1 - \tau) \bm{\phi}',
\end{split}
\end{equation}
where $\tau \ll 1$ is the soft update rate that prevents the target DNN parameters from changing too quickly, which in turn stabilises learning. 



\section{Summary}

This chapter presented the theoretical foundation of the Markov Decision Process (MDP), which serves as the formal mathematical framework for the problem that Reinforcement Learning (RL) aims to solve.
We also explored the field of RL, with particular emphasis on the TD3 algorirthm.
 

Within an MDP, the agent interacts with an environment that is defined by a set of state transition probabilities. 
However, there are many cases (such as autonomous racing) where the state transition probabilities are difficult to represent explicitly.
In these cases, simulators are used to represent them implicitly by generating the state transition according to the current state and selected action.
In the next chapter, we discuss how F1tenth autonomous racing is modelled as a simulator environment, 
such that TD3 can be applied to train RL agents to learn how to race.



